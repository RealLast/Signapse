\documentclass[12pt,a4paper,ngerman,enabledeprecatedfontcommands]{article}
\usepackage[utf8]{inputenc}
\usepackage[german]{babel} 
\usepackage{hyperref}
\usepackage{cleveref}
\usepackage{graphicx}
\usepackage{float}
\renewcommand{\familydefault}{\sfdefault} % hier sieht serifenfreie Schrift ausnahmsweise mal besser aus
\DeclareTextFontCommand{\emph}{\bfseries\em}

\title{Benutzerhandbuch (deutsch)\\Signapse\\Verkehrszeichenerkennung}
\begin{document}
\maketitle

\section{Rechtliches}

\subsection{Haftungsausschluss}
Die App befindet sich im Beta Stadium und dient nur zu Testzwecken. Die Verwendung dieser App geschieht auf eigene Gefahr.\\
Diese App wurde mit größter Sorgfalt entwickelt. Als Herausgeber dieser App können wir jedoch für die Vollständigkeit, Richtigkeit und Verlässlichkeit der bereitgestellten Inhalte und Funktionen keine Haftung übernehmen. Die Nutzung dieser App und ihrer Funktionen erfolgt auf eigene Gefahr. Sie stellt keinen Ersatz für die eigene Aufmerksamkeit dar. Achten Sie stets auf Ihre Umgebung und auf andere Verkehrsteilnehmer und bedienen Sie das Endgerät und die App nicht während der Fahrt.\\

\subsection{Datenschutz}
Die vollständigen Datenschutzbestimmungen befinden sich unter folgendem Link:\\ \url{https://sites.google.com/view/tsd-privacy-policy/deutsch-german-datenschutzerkl\%C3\%A4rung}

\subsection{Verwendete Lizenzen}

\subsubsection{MobileNetV2}
Copyright \textcopyright 2018, Larry\\
MIT Licence

\subsubsection{OpenCV}
Copyright \textcopyright 2000-2018, Intel Corporation, all rights reserved.\\
Copyright \textcopyright 2009-2011, Willow Garage Inc., all rights reserved.\\
Copyright \textcopyright 2009-2016, NVIDIA Corporation, all rights reserved.\\
Copyright \textcopyright 2010-2013, Advanced Micro Devices, Inc., all rights reserved.\\
Copyright \textcopyright 2015-2016, OpenCV Foundation, all rights reserved.\\
Copyright \textcopyright 2015-2016, Itseez Inc., all rights reserved.\\
Third party copyrights are property of their respective owners.\\
BSD Licence

\subsubsection{Single Shot Multibox Detector}
All new contributions compared to the original branch:\\Copyright \textcopyright 2015, 2016, Wei Liu (UNC Chapel Hill), Dragomir Anguelov (Zoox), Dumitru Erhan (Google), Christian Szegedy (Google), Scott Reed (UMich Ann Arbor), Cheng-Yang Fu (UNC Chapel Hill), Alexander C. Berg (UNC Chapel Hill). All rights reserved.\\
All contributions by the University of California:\\Copyright \textcopyright 2014, 2015, The Regents of the University of California (Regents). All rights reserved.\\
All other contributions:\\Copyright \textcopyright 2014, 2015, the respective contributors. All rights reserved.\\

\subsubsection{TensorFlow}
Copyright \textcopyright 2018, The TensorFlow Authors. All rights reserved.\\
Apache License 2.0

\section{Hard- und Softwareanforderungen}

\begin{tabular}{l X[l]}

Betriebssystem: & Android 7.0 \textit{Nougat} (API Version 24) oder höher\\
Prozessor: & Qualcomm Snapdragon 600er/800er-Serie oder vergleichbar\\
Speicher: & 3 GiB Arbeitsspeicher, 512 MiB Massenspeicher frei\\
Sonstiges: & GPS-Empfänger, Kamera mit Bildstabilisierung\\

\end{tabular}
\linebreak

\section{Installation}

\subsection{Installation aus dem Google Play Store}

Laden Sie die App \textit{Signapse} aus dem Google Play Store herunter und befolgen Sie die Anweisungen auf dem Bildschirm.\\
\url{https://play.google.com/store/apps/details?id=de.swp.tsd.trafficsigndetection}

\begin{figure}[H]
\centering
\includegraphics[width=0.5\linewidth]{Benutzerhandbuch/qrcode.png}
\end{figure}

\subsection{Installation via APK}

Um die APK installieren zu können, muss ggf. die Installation von Apps aus unbekannter Herkunft zugelassen werden.

\subsubsection{Android 7.0 \textit{Nougat} bis 7.1.2 \textit{Nougat}}
Die Installation aus unbekannten Quellen kann im in den Geräteeinstellungen erlaubt werden:\\
Einstellungen $\rightarrow$ Sicherheit $\rightarrow$ Unbekannte Herkunft\\
Benutzen Sie eine beliebige Dateimanager-App, navigieren Sie zum Dateipfad der APK der App \textit{Signapse}, installieren Sie diese und folgen Sie den Anweisungen auf dem Bildschirm.

\subsubsection{Android 8.0 \textitt{Oreo} und höher}
Die Installation von Apps ist ab Android 8.0 \textit{Oreo} mit einer Berechtigung der App verbunden, von der sie installiert werden soll. Benutzen Sie eine beliebige Dateimanager-App und erlauben Sie dieser, andere Apps zu installieren. Navigieren Sie anschließend zum Dateipfad der APK der App \textit{Signapse}, installieren Sie diese und folgen Sie den Anweisungen auf dem Bildschirm.

\section{Notwendige Berechtigungen}

Damit die App korrekt funktioniert, muss der Nutzer ihr bis zu zwei Berechtigungen erteilen.

\subsubsection*{Kamera} Diese Berechtigung ist absolut unabdingbar, damit die App funktionieren kann. Falls der Nutzer diese Berechtigung nicht erteilt, kann sie die Kamera nicht benutzen, um Bilder aufzunehmen und dementsprechend auch keine Verkehrszeichen darauf erkennen.

\subsubsection*{Standort} Diese Berechtigung ist optional. Der Nutzer kann diese Berechtigung nicht erteilen, falls er der App \textit{Signapse} nicht erlauben möchte, das GPS zur Geschwindigkeitsermittlung zu benutzen. Falls der Nutzer diese Berechtigung zurückweist, wird die App aber trotzdem noch in der Lage sein, Verkehrszeichen zu erkennen und anzuzeigen.

\section{Betriebsbedingungen}
Die App \textit{Signapse} wird nur Verkehrszeichen der Bundesrepublik Deutschland gemäß StVO §§ 36–43 verlässlich erkennen. Die Funktionalität ist begrenzt auf Verkehrszeichen, welche einen Einfluss auf die aktuell geltende Höchstgeschwindigkeit und/oder auf den Geltungsbereich derselben haben.\\

Falls eine oder mehrere der folgenden Bedingungen erfüllt ist, kann die korrekte Funktion der App \emph{nicht} gewährleistet werden.

\begin{itemize}
    \item Der Nutzer überschreitet eine Geschwindigkeit von 150 km/h
    \item Die Verkehrszeichen wurden nicht gemäß VwV-StVO §§ 39-43 montiert
    \item Nebel, Niederschlag, starkes Gegenlicht
    \item Das Gerät wurde nicht dem Einrichtungs-Menü gemäß kalibriert
    \item Die Windschutzscheibe ist verschmutzt
\end{itemize}

\section{Vor der Benutzung}

Benutzen Sie eine KFZ-Smartphonehalterung, um Ihr Smartphone an der Windschutzscheibe im Querformat zu montieren. Damit das System korrekt funktionieren kann, muss zuvor eine Kalibrierungsphase durchlaufen werden. Dazu werden Sie beim Start der App \textit{Signapse}, nachdem Sie den Haftungsauschuss zur Kenntnis genommen und bestätigt haben, in einem Einrichtungsmenü angewiesen. Befolgen Sie die Anweisungen auf dem Bildschirm und drehen Sie das Smartphone, bis im Kalibrierungsmenü unten rechts ein grüner Haken ($\checkmark$) ist. \\

Um eine gewisse Betriebsdauer gewährleisten zu können, ist es ggf. notwendig, das Smartphone mit einer Stromversorgung zu verbinden - sei es der USB-Anschluss des KFZ, die 12 V-Buchse oder eine Powerbank. Die Stromversorgung muss in der Lage sein, 2 A bei 5 V bereitzustellen. Ein Indiz für eine ausreichende Stromversorgung ist, dass der Ladestand des Akkus während der Verwendung der App \textit{Signapse} nicht sinkt.

\section{Benutzung}
\label{sec:benutzung}

Die App \textit{Signapse} besitzt drei Menüs:

\begin{description}
    \item[Einrichtung] Die zuvor erklärte Einrichtungsphase findet in diesem Menü statt. Der Nutzer kann dieses Menü jederzeit aufrufen oder schließen, falls er eine erneute Kalibrierung für notwendig erachtet. Eine neue Kalibrierung ist insbesondere dann notwendig, wenn der Nutzer unzureichende Detektionsergebnisse bemerkt.  Wird die Einrichtung abgebrochen, bevor das Gerät korrekt kalibriert ist, kann die korrekte Funktion nicht mehr gewährleistet werden.
    
    \item[Detektor] Im Detektor wird die Hauptfunktion der App \textit{Signapse} durchgeführt. In der linken Hälfte wird das zuletzt erkannte Verkehrszeichen sowie die aktuelle Fahrgeschwindigkeit angezeigt; in der rechten Hälfte die Historie der zuletzt erkannten Verkehrszeichen. Die Länge dieser Historie kann im Einstellungsmenü angepasst werden.
    
    \item[Einstellungen] Die Länge der Historie im Detektor-Menü kann hier angepasst werden. Die GPS-Geschwindigkeitsanzeige kann hier an- und abgeschaltet werden.
\end{description}

\section{Fehlerbehebung}

\subsubsection*{Es wird anstelle der Geschwindigkeit eine Tilde angezeigt}
Die Geschwindigkeitsanzeige wurde in der App ausgeschaltet und/oder es wurde die entsprechende Berechtigung nicht gewährt. Schalten Sie die Geschwindigkeitsanzeige in den Einstellungen der App ein und vergewissern Sie sich, dass die App \textit{Signapse} auch Berechtigung für den Standortzugriff besitzt.

\subsubsection*{Es wird immer die Geschwindigkeit 0 km/h angezeigt}
Die Geschwindigkeitsanzeige in der App wurde eingeschaltet und die entsprechenden Berechtigungen wurden gewährt, aber die Standortbestimmung in den Geräteeinstellungen wurde abgeschaltet. Schalten Sie die Standortbestimmung ihres Smartphones ein und starten Sie die App neu.

\subsubsection*{Die Verkehrszeichen werden nur äußerst unregelmäßig erkannt}
Die Betriebsbedingungen sind nicht erfüllt. Vergewissern Sie sich, dass das Gerät korrekt kalibriert ist und überprüfen Sie, ob das Wetter eine korrekte Erkennung der Verkehrszeichen durch die App überhaupt zulässt. Falls letzteres nicht möglich ist, können Sie sie momentan nicht benutzen.

\subsubsection*{Die Verkehrszeichen werden gar nicht erkannt}
Der App wurden nicht die notwendigen Berechtigungen für die Kamera gewährt, oder die Kameralinse des Smartphones ist verdeckt. Erteilen Sie der App \textit{Signapse} die Berechtigung für die Kamera und vergewissern Sie sich, dass die Linse nicht durch die Halterung o.Ä. verdeckt wird.

\subsubsection*{Der Detektor ist nicht verfügbar}
Die der App zugrunde liegende Schnittstelle zu den Neuornalen Netzen konnte nicht initialisiert werden. Installieren Sie die App neu; falls der Fehler danach immernoch auftritt, wenden Sie sich bitte an die Entwickler der App.

\end{document}