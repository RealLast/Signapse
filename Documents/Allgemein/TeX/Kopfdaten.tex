\usepackage[left=2.50cm, right=2.50cm, bottom=2.50cm, top=2.50cm]{geometry}
\usepackage[utf8]{inputenc}
\usepackage[german]{babel} 
\usepackage{natbib}
\usepackage{graphicx}
\usepackage[hyperfootnotes=false, hidelinks]{hyperref}
\usepackage[toc,section=section, numberedsection, nonumberlist]{glossaries}
%\usepackage{tabularx}
\usepackage{tabu}
\usepackage{listings}
\usepackage[nameinlink]{cleveref}
\usepackage{float}
\usepackage[compact]{titlesec}
\usepackage[table]{xcolor}
\usepackage{svg}
\usepackage{titling}
\usepackage{nameref}
\usepackage{wasysym}
\usepackage{pifont}
\usepackage{verbatim}
\usepackage{textcomp}
\usepackage{stmaryrd}
\usepackage{etoolbox}
\usepackage{xpatch}
\usepackage{siunitx}
\usepackage{stringstrings}
\usepackage{chngcntr}
\usepackage[all]{hypcap}
\usepackage{pdfpages}
\usepackage{tablefootnote}
\usepackage{lscape}
\usepackage{eurosym}
\usepackage{xcolor}
%\usepackage{comment}

% Erlaubt redefinieren dieser Befehle, behebt Konflikte
\makeatletter
    \let\diameter\relax
    \let\leftmoon\relax
    \let\rightmoon\relax
    \let\newmoon\relax
    \let\fullmoon\relax
\makeatother
\usepackage{mathabx}

% Hebt Kapiteltitel
\setlength{\droptitle}{-10em}

% Macht Fußnoten nach Zeilenumbruch linksbündig
\deffootnote[1em]{1em}{1em}{\textsuperscript{\thefootnotemark}}

% nummeriert im Inhaltsverzeichnis nur bis subsection
\setcounter{tocdepth}{2}
\setcounter{secnumdepth}{2}

%\hypersetup{
%    linkcolor=red,
%    urlcolor=black
%}

% streckt Tabellenzellen
\tabulinesep =3pt

% Verhindert, dass Fußnoten pro Kapitel von vorne nummeriert werden
\counterwithout*{footnote}{chapter}

\lstset
        {
            basicstyle=\small\ttfamily,
            breaklines=true,
            backgroundcolor = \color{gray!10},
            xleftmargin = 0.48cm,
            framexleftmargin = 1em
        }
        
\titleformat{\chapter}{\vspace{-2.5cm}\bf\huge}{\thechapter.}{20pt}{\bf\huge}

\renewcommand{\labelitemi}{$\RHD$}
\renewcommand{\labelitemii}{$\blacktriangleright$}

\addtokomafont{disposition}{\rmfamily}

\addtokomafont{descriptionlabel}{\rmfamily}

\setlength{\fboxsep}{0pt}%
\setlength{\fboxrule}{1pt}

\setlength{\parindent}{0pt}

\sisetup{   locale=DE,%
            round-mode=places,%
            round-precision=2,%
            per-mode=fraction%
}

\let\texteuro\euro

\DeclareSIUnit[per-mode=fraction]\schild{Schilder}
\DeclareSIUnit[per-mode=fraction]\PS{PS}

% Plus and Minus
\def\Plus{\texttt{+}}
\def\Minus{\texttt{-}}

% Table Highlights
\def\hl{\cellcolor[HTML]{DDDDDD}}

\let\subsubsubsection\paragraph

% Checkmark
\newcommand{\cmark}{\ding{51}}

\newcommand{\twodigit}[1]{\ifnum#1<10 0#1\else#1\fi}

% Links
\newcommand{\link}[1]{\href{#1}{#1}}
\newcommand{\footlink}[2]{\footnote{\href{#1}{\text{#1}}, #2}}
\newcommand{\footlinklabel}[3]{\footnote{\label{#3}\href{#1}{\text{#1}}, #2}}

\newcommand{\footlinktext}[2]
{
    \stepcounter{footnote}
    \footnotetext{\href{#1}{\text{#1}}, #2}
}

\newcommand{\footrealtext}[1]
{
    \stepcounter{footnote}
    \footnotetext{\text{#1}}
}

\newcommand{\linebar}
{
    \par\noindent\rule[\baselineskip]{\textwidth}{0.4pt}
}

%%%%%%%%%%%%%%%%%%%%%%%%%%%%%%%%%%%%%%%%%%%%%%%%%%%%%%%%%%%%%%%%%%%%%%%%%%%%%%%%%%%%%
% MuSCoW Naming                                                                     %
%%%%%%%%%%%%%%%%%%%%%%%%%%%%%%%%%%%%%%%%%%%%%%%%%%%%%%%%%%%%%%%%%%%%%%%%%%%%%%%%%%%%%

%\newcommand{\must}[1]{$\llbracket$#1$\rrbracket$}
%\newcommand{\should}[1]{$[$#1$]$}
%\newcommand{\could}[1]{$($#1$)$}
%\newcommand{\wont}[1]{$!$#1$!$}

\newcounter{musts}
\setcounter{musts}{0}
\newcounter{shoulds}
\setcounter{shoulds}{0}
\newcounter{coulds}
\setcounter{coulds}{0}
\newcounter{wonts}
\setcounter{wonts}{0}

\newcounter{chaprequirements}
\setcounter{chaprequirements}{0}

\newcommand*{\currentchapter}{Not set yet.}

\makeatletter
\xpretocmd{\@chapter}
{%
    \renewcommand*{\currentchapter}{#1}%
    \setcounter{chaprequirements}{0}%
}{}{}
\makeatother

\makeatletter
\xpretocmd{\section}
{%
    \setcounter{chaprequirements}{0}%
}{}{}
\makeatother

\newcommand*{\chaplet}{\substring[v]{\currentchapter}{1}{1}}
\newcommand*{\chapnum}{\substring[v]{\thesection}{1}{1}}
\newcommand*{\secnum}{\substring[v]{\thesection}{3}{3}}
\newcommand*{\reqcount}{\twodigit{\thechaprequirements}}

\newcommand*{\IDcode}{\chaplet\chapnum\secnum\reqcount}

\newcommand*{\musttext}{$\llbracket$\IDcode$\rrbracket$}
\newcommand*{\shouldtext}{$[$\IDcode$]$}
\newcommand*{\couldtext}{$($\IDcode$)$}
\newcommand*{\wonttext}{$!$\IDcode$!$}

\newcommand*{\must}
{%
    \refstepcounter{musts}%
    \stepcounter{chaprequirements}%
    \musttext%
    %\label{#1}%
}

\newcommand*{\should}
{%
    \refstepcounter{shoulds}%
    \stepcounter{chaprequirements}%
    \shouldtext%
    %\label{#1}%
}

\newcommand*{\could}
{%
    \refstepcounter{coulds}%
    \stepcounter{chaprequirements}%
    \couldtext%
    %\label{#1}%
}

\newcommand*{\wont}
{%
    \refstepcounter{wonts}%
    \stepcounter{chaprequirements}%
    \wonttext%
    %\label{#1}%
}

\newcommand*{\mustref}[1]
{%
    $\llbracket$#1$\rrbracket$%
}

\newcommand*{\shouldref}[1]
{%
    $[$#1$]$%
}

\newcommand*{\couldref}[1]
{%
    $($#1$)$%
}

\newcommand*{\wontref}[1]
{%
    $!$#1$!$%
}

\newcommand{\accref}[1]
{%
    $/$#1$/$%
}

%\crefname{musts}{}{}
%\creflabelformat{musts}{%
%   \addtocounter{chaprequirements}{-1}%
%    \musttext%
%    intern: #1
%    \addtocounter{chaprequirements}{1}%
%}
%
%\crefname{shoulds}{}{}
%\creflabelformat{shoulds}{%
%    \addtocounter{chaprequirements}{-1}%
%    \shouldtext%
%    intern: #1
%    \addtocounter{chaprequirements}{1}%
%}
%\crefname{coulds}{}{}
%\creflabelformat{coulds}{%
%    \addtocounter{chaprequirements}{-1}%
%    \couldtext%
%    intern: #1
%    \addtocounter{chaprequirements}{1}%
%}
%\crefname{wonts}{}{}
%\creflabelformat{wonts}{%
%    \addtocounter{chaprequirements}{-1}%
%    \wonttext%
%    intern: #1
%    \addtocounter{chaprequirements}{1}%
%}

\makenoidxglossaries
% https://alphabetizer.flap.tv/

\newglossaryentry{API}{name={API}, description={Application Programming Interface.\\ Programmierschnittstelle}}

\newglossaryentry{App}{name={App}, description={Application.\\Programm, welches auf einem Smartphone ausgeführt werden kann. Hier im Besonderen: Ein Teil des von uns entwickelten Produkts als Benutzerschnittstelle}}

\newglossaryentry{ALU}{name={ALU}, 
description={Arithmetic Logic Unit.\\ Das elektronische Rechenwerk eines Prozessors. Es ist speziell konzipiert um mathematische Operationen hocheffizient und schnell durchzuführen}}

\newglossaryentry{ARM}{name={ARM},
description={Acorn RISC Machine, bzw. Advanced RISC Machine.\\ Eine weit verbreitete Mikroprozessorarchitektur, welche sehr oft innerhalb von \gls{SoC}s in \gls{Smartphone}s zum Einsatz kommt}}

\newglossaryentry{CPU}{name={CPU}, 
description={Central Processing Unit.\\ Zentrales Rechen- und Steuerwerk einer Rechenmaschine, hier insb. eines Personalcomputers oder eines Smartphones}}

\newglossaryentry{Data Member}{name={Data Member}, description={~\\Klasseneigenschaft bzw. Membervariable. Eine Variable, die einer Instanz einer Klasse zugeordnet ist und von Instanz zu Instanz variieren kann}}

\newglossaryentry{Deep Learning}{name={Deep Learning}, description={~\\Klasse von Optimierungsmethoden Neuronaler Netzwerke, die zahlreiche Zwischenlagen zwischen Eingabeschicht und Ausgabeschicht haben und dadurch eine umfangreiche innere Struktur besitzen}}

\newglossaryentry{Drehachsen}{name={Drehachsen}, description={~\\Achsen, um die das Smartphone gedreht werden kann. Es gibt die Roll-, Gier- und Nickachse. \begin{figure}[H]
\centering
\includegraphics[width=0.5\linewidth]{Reviewdokument/Grafiken/drehachsen.png}
\end{figure}}}

\newglossaryentry{Detektion}{name={Detektion}, description={~\\Lokalisierung eines Merkmals (insb. eines Verkehrszeichens) in einem Bild}}

\newglossaryentry{Fahrzeug}{name={Fahrzeug}, description={~\\Handelsüblicher Personenkraftwagen (PKW), insbesondere kein Lastkraftwagen\\ (LKW)}}

\newglossaryentry{Filter}{name={Filter}, description={~\\Ein Verarbeitungsschritt. Jeder Filter hat eine Dateneingabe und eine -ausgabe. In jedem Verarbeitungsschritt werden die einkommenden Daten umgewandelt. Bei der Umwandlung können den Daten Teile entnommen, hinzugefügt oder auch vollständig ersetzt werden. Die Art der Umwandlung wird durch den Filter bestimmt}}

\newglossaryentry{Filterverwaltungs-Bibliothek}{name={Filterverwaltungs-Bibliothek}, description={~\\Programm-Bibliothek, welche Möglichkeiten bietet, Filter und Neuronale Filter zu laden und auf Bildserien anzuwenden}}

\newglossaryentry{FLOPs}{name={FLOPs}, 
description={Floating Point Operations.\\Operationen auf Fließkommazahlen, die von der \gls{ALU} ausgeführt werden. Sie gelten als eine der aufwendigsten, elementaren Operationen und bestimmen maßgeblich die Laufzeit eines Programms. Ausdrücklich zu unterscheiden von \glqq FLOPS\grqq}}

\newglossaryentry{FLOPS}{name={FLOPS}, 
description={Floating Point Operations per Second.\\Maß für die Leistungsfähigkeit von Computern. Ausdrücklich zu unterscheiden von \glqq FLOPs\grqq}}

\newglossaryentry{Geschwindigkeitsschild}{name={Geschwindigkeitsschild}, description={~\\Gibt die zulässige Höchstgeschwindigkeit des Fahrzeugs an, an dessen Rückseite es befestigt ist. Siehe §58 \gls{StVZO}}}

\newglossaryentry{GPU}{name={GPU}, 
description={Graphics Processing Unit; Grafikkarte.\\ Rechen- und Steuerwerk einer Rechenmaschine, hier insb. eines Personalcomputers oder eines Smartphones, welches auf die Bildverarbeitung spezialisiert ist}}

\newglossaryentry{Keras}{name={Keras}, description={~\\Deep-Learning-Framework in Python basierend auf TensorFlow}}

\newglossaryentry{Klassifikation}{name={Klassifikation}, description={~\\Einteilung eines Merkmals in eine Klasse von Urbildern}}

\newglossaryentry{Konfidenz}{name={Konfidenz}, description={~\\Sicherheit, dass das Objekt korrekt klassifiziert wurde}}

\newglossaryentry{Label}{name={Label}, description={~\\ Meta-Informationen zu einem Sample. In diesem Zusammhang die Position, Größe und Art von Verkehrsschildern in einer Bildaufnahme}}

\newglossaryentry{mAP}{name={mAP}, description={mean Average Precision.\\Maß für die Genauigkeit eines Objekt-Detektors und -Klassifikators. Basiert auf der Position und Größe des vermuteten Bildausschnitts sowie der Rangverteilung der vermuteten Klasse}}

\newglossaryentry{Multithreading}{name={Multithreading}, description={Nebenläufigkeit.\\ Gleichzeitiges Abarbeiten mehrerer Threads innerhalb eines Prozesses}}

\newglossaryentry{Neuronaler Filter}{name={Neuronale Filter}, description={~\\Ein Filter, dessen Umwandlungsvorgehen durch ein Neuronales Netz bestimmt wird}}

\newglossaryentry{Neuronales Netzwerk}{name={Neuronales Netzwerk}, description={~\\Hier im Besonderen: Künstliches Neuronales Netzwerk. Simplifiziertes Modell eines Nervensystems, bestehend aus künstlichen Neuronen, welche auf Computern trainiert werden, um komplexe Aufgaben im Bereich der künstlichen Intelligenz zu bewältigen}}

\newglossaryentry{Nutzer}{name={Nutzer}, description={~\\Der aktuelle Fahrzeugführer, während sich das Fahrzeug nicht in Fahrt befindet bzw. der Beifahrer, während sich das Fahrzeug in Fahrt befindet. Es wird das generische Maskulinum verwendet}}

\newglossaryentry{OpenCV2}{name={OpenCV2}, description={Open Source Computer Vision Library. \\Quelloffene Computergrafikbibliothek. Hier verwendet um Datensätze sowie das erfasste Kamerabild auf die Eingabespezifikationen der Neuronalen Netzwerke anzupassen}}

\newglossaryentry{Produkt}{name={Produkt}, description={~\\Die Kombination aus App und API, welche im Rahmen dieses Softwareprojektes entwickelt werden}}

\newglossaryentry{Pipe}{name={Pipe}, description={~\\Eine Pipe stellt eine Verbindung zwischen den einzelnen Verarbeitungsschritten dar}}

\newglossaryentry{Pipes and Filters}{name={Pipes and Filters}, description={auch Datenfluss-System.\\ Ein Architekturmuster welches die Struktur für Systeme, die Datenströme verarbeiten, darstellt}}

\newglossaryentry{Protobuf}{name={Protocol Buffers, Protobuf}, description={~\\Bezeichnet ein von Google entwickeltes, programmiersprachen- und plattformunabhängiges Format sowie die dazugehörige Implementierung für die Serialisierung und Deserialisierung strukturierter Daten.}}

\newglossaryentry{Sample}{name={Sample}, description={~\\In diesem Zusammenhang eine Bildaufnahme von Verkehrszeichen, welche Neuronalen Netzwerken präsentiert werden können, um sie zur Detektion und Klassifkation zu trainieren}}

\newglossaryentry{Smartphone}{name={Smartphone}, description={~\\Mobiles Endgerät, welches über eine integrierte Kamera verfügt und den im Pflichtenheft spezifizierten Bedingungen genügt}}

\newglossaryentry{SoC}{name={SoC}, description={System-on-a-Chip.\\ Kombination von CPU, Bus, Taktgeber, Co-CPUs, GPU, Soundchip, Interfaces, etc. oder Teilen davon auf einem Chip. Häufig verwendet in mobilen Endgeräten und eingebetteten Systemen um durch hohe Integrationsdichten Fläche auf der Platine einzusparen}}

\newglossaryentry{StVO}{name={StVO}, description={Straßenverkehrsordnung\footlink{https://www.gesetze-im-internet.de/stvo\_2013/StVO.pdf}{29.04.2018}}}

\newglossaryentry{StVZO}{name={StVZO}, description={Straßenverkehrszulassungsordnung\footlink{https://www.gesetze-im-internet.de/stvzo\_2012/StVZO.pdf}{29.04.2018}}}

\newglossaryentry{System}{name={System}, description={~\\Gesamtheit aus Halterung, Smartphone und App}}

\newglossaryentry{Tracking}{name={Tracking}, description={~\\Objekte anhand primitiver Merkmale in einer Bildserie oder einem Video lokalisieren und verfolgen}}

\newglossaryentry{TensorFlow Lite}{name={TensorFlow Lite}, description={~\\Für mobile Endgeräte optimiertes Deep-Learning-Framework des Google Brain Teams\footlink{https://www.tensorflow.org/mobile/tflite/}{29.04.2018}}}

\newglossaryentry{TensorFlow Mobile}{name={TensorFlow Mobile}, description={~\\Deep-Learning-Framework des Google Brain Teams, welches zwar für Android noch keine GPU Unterstützung bietet, dafür aber derzeit noch bzgl. der Kompatibilität besser von bereits etablierten Netzwerken unterstützt wird}}

\newglossaryentry{Tensorflow Node}{name={Tensorflow Node},description={~\\Ein Knoten (Node) im Sinne von TensorFlow bezeichnet eine bestimmte Operation innerhalb des Netzwerk Models: Es ist die grundlegende Einheit für Berechnungen innerhalb von TensorFlow. Die Ergebnisse dieser Knoten lassen sich anhand deren Bezeichnungen (zum Beispiel \glqq{}final\_output\qrqq{})  einzeln abfragen}}

\newglossaryentry{UI}{name={UI},description={User Interface.\\ Benutzeroberfläche}}

\newglossaryentry{VZ}{name={VZ}, description={Verkehrszeichen}}

\newglossaryentry{Verkehrszeichen-API}{name={Verkehrszeichen-API}, description={~\\Besondere Programmierschnittstelle, die für Verkehrszeichenerkennung optimiert ist und erweiterte Logik zur Verfügung stellt, wie etwa die Abschätzung, wann ein Verkehrsschild passiert wurde und damit gültig ist und angezeigt werden muss}}

\newglossaryentry{Verkehrszeichenkombination}{name={Verkehrszeichenkombination}, description={~\\Gruppe von untereinander angebrachten Verkehrszeichen, welche einen gemeinsamen Sinnzusammenhang darstellen}}

\newglossaryentry{VwV-StVO}{name={VwV-StVO}, description={Allgemeine Verwaltungsvorschrift
zur Straßenverkehrs-Ordnung\footlink{http://www.verwaltungsvorschriften-im-internet.de/bsvwvbund\_26012001\_S3236420014.htm}{29.04.2018}}}



\author{
    [Projektgruppe 2]\\~\\
    \textit{Namen entfernt}\\~\\
    \textcolor{red}{Verwendung nur zu Lehrzwecken an der TU Ilmenau. Die Verwendung}\\\textcolor{red}{für private oder kommerzielle Zwecke ohne Einwilligung der Autoren ist}\\ \textcolor{red}{ausdrücklich untersagt.}\\~\\
    Kontakt: \href{mailto:signapse.app@gmail.com}{signapse.app@gmail.com}
    %Bock, Robert Niklas\\ \texttt{robert-niklas.bock@tu-ilmenau.de}\\~\\
    %Dirbas, Mohammad\\ \texttt{mohammad.dirbas@tu-ilmenau.de} \\~\\
    %Fischedick, Söhnke Benedikt\\ \texttt{soehnke-benedikt.fischedick@tu-ilmenau.de} \\~\\
    %Hampel, Jakob Frank\\ \texttt{jakob.hampel@tu-ilmenau.de} \\~\\
    %Köhler, Florian\\ \texttt{florian.koehler@tu-ilmenau.de} \\~\\
    %Langer, Patrick\\ \texttt{patrick.langer@tu-ilmenau.de} \\~\\
    %Nowati, Yusuf\\ \texttt{yusuf.nowati@tu-ilmenau.de} \\~\\
    %Treichel, Tim\\ \texttt{tim.treichel@tu-ilmenau.de} \\ ~\\ ~\\ ~\\ ~\\
}
